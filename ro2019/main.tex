% start of document preamble

%%%% Proceedings format for most of ACM conferences (with the exceptions listed below) and all ICPS volumes.
\documentclass[sigconf,screen,nonacm]{acmart}
%\documentclass[sigconf]{acmart}

%\acmConference[TaPP]{Theory and Practice of Provenance}{June 3, 2019}{Philadelphia, PA}
%\setcopyright{rightsretained}

% defining the \BibTeX command - from Oren Patashnik's original BibTeX documentation.
\def\BibTeX{{\rm B\kern-.05em{\sc i\kern-.025em b}\kern-.08emT\kern-.1667em\lower.7ex\hbox{E}\kern-.125emX}}
 
\usepackage[english]{babel}
\usepackage{enumitem}
\setlist[itemize]{leftmargin=3 mm}

% start of the body of the document body.
\begin{document}

% title
\title[Reproducibility by Other Means]{Reproducibility by Other Means: Transparent Research Objects with Science-Oriented Provenance}

\author{Timothy McPhillips  \qquad Bertram Lud\"ascher}
% BL: The braces don't seem to work ... 
%\email{\{tmcphill,lanl2,nnp2,ludaesch\}@illinois.edu}
 
% BL: so let's do some manual hacking: 
\affiliation{%
  \institution{School of Information Sciences,  University of Illinois at Urbana-Champaign\\ 
    \texttt{\{tmcphill,ludaesch\}@illinois.edu} }
}

% This command processes the author and affiliation and title information and builds the first part of the formatted document.
\maketitle



\textbf{ABSTRACT}

\noindent Research Objects have the potential to significantly enhance the reproducibility
and transparency of scientific research.  One important way Research Objects can do this
is by encapsulating the means for re-executing the computational components
of studies, thus supporting the new form of reproducibility enabled by digital 
computing---exact repeatability.  However, Research Objects also can make 
scientific research more reproducible by supporting transparency, a component
of reproducibility orthogonal to re-executability.  We describe here our vision for 
making Research Objects more transparent by providing means for disambiguating 
claims about reproducibility generally, and computational repeatability specifically.
We show how support for science-oriented queries can enable researchers to
assess the reproducibility of Research Objects and the individual methods and results
they encapsulate.

%%% Local Variables: 
%%% mode: latex
%%% TeX-master: "main"
%%% End: 

\section{Introduction}

	The foundation of modern science  is the expectation that the observations, experiments, and
		predictions that comprise scientific research be independently verifiable by others.
	This requirement often is referred to as reproducibility (or replicability).
	Digital computing approaches make it possible to repeat exactly certain computational aspects of research
		to an extent that exceeds what can be achieved when observing or experimenting with natural 
		phenomena in the physical world.  
	It generally is expected that computational processes, the implementation of hardware and software 
		enabling those processes, and the outputs of those processes all can be repeated exactly.
	Where computing makes up a significant fraction of scientific research, the terms reproducibility and
		replicability often are associated with this rather different expectation. 
	Moreover, while this expectation may be realizable in principle, in practice the complexities of real-world
		hardware and software make this kind of reproducibility very challenging to achieve except in very limited cases.
	Currently much effort is going into expanding the fraction of scenarios in which the computational
		components of research can be repeated exactly over ranges of time and space relevant to scientific 
		research and discourse.
	These efforsts are important for the research community to pursue, and for science funding
		agencies to support, especially because the computing industry generally does not have 
		requirements for exact repeatability across signficant spans of time.
	However, it also is crucial to note that the concept of exact repeatability of the kind pursued by these efforts is 
		qualitatively different from the concept of reproducibility (or replicability), that underlies the natural sciences.
	It is even more important to realize that scientific reproducibility generally is not simply a weaker form of computational
		replicability.  
	The relationship between replicability and reproducibility as defined here is not one of degree.
	In particular, achieving computational replicability does not automatically deliver scientific reproducibility.
	This is both bad and good news.
	The bad news is that it is possible to put much effort into achieving computational replicability without delivering
		the kind of reproducibility that is critical for producing trustworthy science.
	The good news is that scientifically-meaningful reproducibility can be realized in cases (or over spans of time)
		where computational replicability is either impossible or impractical given existing technology.
	Reseachers in the natural sciences are comfortable with the idea that it may not be possible to exactly
		repeat all reported observations, procedures, and experimental results.
	They do not see this concession to real-world practicalities as a contradiction to their demand that science be reproducible.
	What the natural sciences actually demand is that:
		(a) research procedures be repeatable by others in principle
		(b) the means of repeating the work be subject to review and evaluation
		(c) this review and evaluation be possible without actually repeating the work

Challenges of reproducibility

	Challenges of terminology: Reproduce vs Replicate

		The difference between them

		Standardization proposals

		History of usage in biology

			Cell and molecular biologists study both replication and reproduction as natural processes.
			DNA replicates :  high fidelity, variation not desired, ingredients indistiguishable, errors are corrected on the fly.
			Organisms reproduce:  lower fidelity expected, variation desired, different ingredients acceptable.
			Cells have replisomes, complex molecular machines where DNA replication occurs, and copying errors are detected and corrected.
			In origin of life research a crucial debate is over 'replication first' (DNA World) or 
 					'metabolism first' (aka reproduction first, i.e. without replicating genetic material).
			These terminologies with biology are well established.
			Biologists also have a rich and well-defined vocabulary to describe replicability of experimental measurements and results.
			Commonly distinguish two kinds of experimental 'replicates':  technical replicates, and biological replicates.
			FASEB definitions of reproducibility and replicability are consistent with the above.
			Footnote: provide number of FASEB members (105), number of societies and subdisciplines (29) represented by FASEB.
 			Reality is that the terminology is well established in large branches of science already.

		Need for reproduction/replication to mean different things in different fields

			The relationships of corresponding concepts across fields is one of analogy, not identity.
			Exact repeatability is at least theoretically possible and sometimes practical under realistic assumptions when an experiment 
				is entirely in silico and isolated from the outside world.
			As soon as observation of the real world is involved, exact repeatability often is impossible.
			Neither of the types of replicates in experimental biology are exact, although both are measures of repeatability.
			There often is no way to repeat exactly an experiment that involves scientific instruments, physical samples, or experimental apparatus.
			In contrast, it is not unreasonable to talk about exactly repeating a purely computational experiment, at least by the original researcher,
				on the same hardware, close in time to the original experiment.
			In reality, computational repeatability is not as easy as sometimes assumed (see below), but fundamentally this is
				a different situation than when a scientific instrument is involved or observations are made of the external world.

		Our approach to terminology

			Respect differences between fields of research and different expectations with regard to reproducibility.
			Do not expect standardization to even be meaningful (never mind politically achievable) across domains.
			Instead, only use the R* words in ways that make their meanings clear in context.
			Do not be surprised if computational sciences turn out not to be representative of science generally.

		Specific implications for our contributions to the Research Objects field

			Take care to define R* words precisely when expressing desiderata, describing features, or making or comparing claims about capabilities. 
			Do not expect efforts to achieve computational repeatability alone to enable "reproducible science" generally.

	Challenges of computational replicability
	
		The fundamental limitations computers impose on replicability of program executions are well known.
	
			Finite precision arithmetic, different word sizes on different processors, round-off errors, etc, impose limits on scientific computations and 
				their replicability across different computing environments.
			Virtual machines and containers do not address these issues. Full emulation is required to run the same binary in identical fashion
				on a different processor.  This is typically slow.
			These limitations are even more challenging to manage reproducibly because programs typically are compiled, meaning that the
				exact sequence of machine instructions executed even by a single processor cannot generally be controlled.  A different compiler,
				or a newer version of the same compiler will yield a different sequence of machine instructions.

		Replicating the outputs of a program is far from straightforward

			Observing that a program or set of programs can be executed again is not sufficient to conclude that the underlying computation
				was replicated.  The outputs of the programs must be checked for equivalence.
			Because of the expected variation in run time behavior of programs due to the issues above, checking that outputs of a program
				run are equivalent to the outputs of a run of that program is not always as simple as 
				comparing the outputs for bitwise identity.
			Robustly checking for equivalence of output generally must be confirmed in some way other than comparing files at the bit level.
			Footnote: The excellent practice of including accurate provenance and other meaningful metadata in data file headers makes it 
				even more unlikley that outputs from different runs will be bitwise-identical.
	
		Replicating just the software running the program is challenging in practice
	
			How can we ensure that the stream of instructions sent to the processor for two executions is identical?
			Even holding the computer hardware and compiler version constant, programs depend on language libraries, OS libraries, and system calls.
			Much scientific software also depends directly and indirectly on large numbers of 3rd-party libraries.
			Only direct dependencies can be controlled reliably at build-time.  And many dependencies are via shared libraries that can change between executions
				of the exact same executable--no recompile is needed to get a new effective executable.
			Footnote:  Fans of the Go programming language are bringing back the static executable.
			Recompiling or even just rerunning the "same" program a week later can result in a completely different effective instruction stream.
			
		Even reproducing computing environments is hard
	
			Containers and their discontents
				Footnote:  By 'discontents' we do not mean that we object to the use of containers, but rather than we are not content with container technology alone.
				There currently is much enthusiasm around containers as a means of reproducing computing environments and making computational science replicable.
				Whole Tale is one of several projects leveraging the capabilities of containers for this purpose.  Others include Binder and Code Ocean.
				In Whole Tale it is recognized that containers alone cannot satisfy researcher's needs for sharing their computing environments and computations.
				Rather, container technology such as Docker provide an invaluable tool for the reproducible science software stack architect.
				A major motivation for funding (and continuing to fund) projects like Whole Tale is that the containers on their own are insufficient as means
					to making computational science reproducible, and it is not practical for individual researchers and groups to use containers 
					and other technology to actually achieve scientifically meaningful reproducibility over periods longer than the publication-cycle time scale.
	
			What containers do not do
				Despite what sound like suggestions to the contrary in the literature, container technology such as Docker do not ensure computational replicability,
					and do not on their own solve any of the problems of computational replicability described above.
				What containers do provide a very convenient means for executing customized computing environments on behalf of researchers, withot having to run
					an entire virtual machine for each environment.
				In common with virtual machine technology, containers do not abstract or hide the underlying hardware architecture of the computer on which they run.
				They do not abstract the underlying operating system, but simply use the Linux kernel on the host.  Kernel parameter settings on the host apply to 
					all containers running on the host (reference famous blog post on the topic "Containers Do Not Contain").
				Rebuilding an image from its Dockerfile specification is not guaranteed to yield the same image.  It general it will not.
				Container images, once they are built, are not guaranteed to run on future releases of the container host.
				They also do not ensure that computations run within the container will replicable in the feature.
	
			What containers are for
				What containers are good for is precisely what they were to do for the computing industry:  enable developers to write and test code in a
					computing environment of a developer's choosing that can then be replicated on a very short time scale (hours or days) in staging 
					and production environments.
				Containers also are good at managing conflicts in dependencies between different components of a multiprocess software architecture.
				Using containers to 'contain' dependencies in this way is most effective when an application can be split across multiple containers running in concert. 
				The model of one container, one computing environment does not lend itself to dependency isolation. 
	
			The problem of time and dependencies
				An emerging threat to reproducibility of computational science is the spread misconception that sharing the definition of a container image, 
					e.g. by including a Dockerfile in the Git repo for the project, is a guarantee that others (or even the original researcher) will be able to 
					recreate the corresponding image and computing environment it represents.
				Researchers making this assumption may be less likely to preserve all of the information actually required to reproduce their computations.
				A major reason a Dockerfile is not enough is that the implicit dependencies of the built environment are constantly changing.
				This is well known to anyone working directly with Docker, or other container technologies.
				Here we will give a single example of the implications of this issue for reproducible science.
	
	What is reproducibility really for?
	
		Achieving meaningful replicability even for the computational parts of research is very challenging.
		But this is no reason to give up hope.  
		Replicability is a means to an end--justification of scientific results--and Research Objects can help us achieve that end by other means.
		What is most exciting about Research Objects is that they can achieve this end despite the difficulty of computational replicability.
		And the primary means by which Research Objects can due this is by providing transparency.
\section{Reproducibility in science}

Modern science is founded on the expectation that the observations, experiments, and
	predictions that comprise scientific research be independently verifiable by others.
This requirement, referred to as the \emph{reproducibility} or \emph{replicability} of 
	science, applies not only to the products of research studies (\emph{substances}, 
	\emph{results}, \emph{conclusions}, \emph{models}, \emph{data products}, 
	\emph{predictions}), but also to the activities that ultimately give rise to these
	products (\emph{methods}, \emph{protocols}, \emph{workflows}); the materials 
	employed in these activities (\emph{reagents}, \emph{instruments}, 
	\emph{software}); and the conditions 
	under which which these activities are carried out (\emph{temperatures},
	\emph{instrument settings}, \emph{software parameters},
	\emph{computing environments}).  When sufficient details are available
	such that the research products and methods can be reviewed, interpreted, and
	evaluated by other researchers \emph{without} repeating the work, a study is said to be 
	\emph{transparent}.

While it is true that studies attempting primarily to reproduce previous results are relatively rare in the
	pure natural sciences, even the most groundbreaking studies in these fields include components	
	that explicitly or implicitly confirm the reproducibility of previously reported results and procedures.
The expectation is that new studies will reliably produce meaningful results consistent with previous work 
	only if the prior work on which they are based or otherwise relates to is reproducible.
In this sense, the whole of basic research in the natural sciences can be seen as an ongoing, massively-parallel
	reproducibility study that also happens to produce a steady stream of new results.
Exceptions to this pattern occur when studies appear to overturn well-established understandings of nature,
	violate the expectations of how research in a particular field is to be carried out, or otherwise cause controversy.
In these cases direct attempts may be made to reproduce results by duplicating as carefully as possible
	the reported methods and conditions described in the controversial study.

Even when attempts are made specifically to confirm the reproducibility of particular studies or results, investigators in
	the natural sciences generally do not expect the processes and products of research to be duplicated exactly.
The vast majority of quantitative observations made of real world phenomena using scientific instruments
	are associated with limited precision and other intrinsic uncertainties that must themselves be characterized
	and well understood for science based on them to be considered reproducible.
It is a hallmark of trustworthy science that quantitative observations and claims are inseparable from these 
	uncertainties in measurement and their propagation through data analysis.

Similarly, the materials and processes employed in the natural sciences generally are impossible
	to duplicate exactly.
In a chemistry laboratory, the precise quantities of input reagents will vary, temperatures will differ, and heating
	or cooling rates will be unique for each run of a chemical synthesis, no matter how carefully these conditions
	are controlled; the yield and purity of the intended product necessarily will vary as well from run to run.
A similar situation holds when measurements are made on samples using a scientific instrument. 
Different instruments of the same model will vary slightly and produce slightly different results even
	under identical conditions on identical samples.
Generally, the original researchers are in the best position to assess how the minimum variation expected
	between runs of a synthesis (they have access to the same batch of reagents and the same equipment), 
	or between repeated readings of an instrument on the same or equivalent samples (they can prepare
	multiple samples at the same time, and run these samples through the instrument one after the other).
A researcher attempting to duplicate another's work can expect to see greater deviation from the reported results	
	because the materials and conditions involved will necessarily differ to a greater degree.

This asymmetry between the original researcher and another repeating the work is reflected in the distinction
	between replicability and reproducibility in experimental biology.
Replication is something that long and rich literature
	exploring the importance of \emph{technical replicates} and \emph{biological replicates} in experimental biology;
	the former refer to repeated measurements performed on the same sample, the latter to measurements made
	to different but equivalent samples.

FOOTNOTE: A significant fraction of high-performance computing resources worldwide are dedicated to analyzing the vast quantities of
	experimental data produced by Next-Generation Sequencing (NGS) methods where replicates of this kind are essential for
	assessing the quality of the data input to these analyses. Genomics is just one field where the terminologies surrounding
	reproducibility in the natural sciences and for digital computing must inevitably collide.

In contrast, digital computing approaches make it possible to repeat \emph{exactly} certain computational aspects of research,
	 even by \emph{different} researchers using \emph{different} computers.  
Indeed, it generally is expected that computational processes, the implementation of hardware and software 
	enabling those processes, and the outputs of those processes all can be repeated exactly by others--at least in principle.
This potential of exact repeatability is unquestionably of enormous value to any field of research employing computers,
	and certainly will contribute to the ability of researchers in every field to reproduce or build on others' work.
At the same time, there is at least some risk of this new expectation of exact repeatability being conflated 
	(consciously or unconsciously) with the longstanding understanding of reproducibility in the basic sciences. 
It is essential that the new concept be kept distinct.

Moreover, while computational experiments and analyses may be exactly repeatable in principle, 
	in practice the complexities of real-world hardware and software currently make computational repeatability 
	challenging to achieve in practice except over very limited time scales.
Because of the obvious value that exact repeatability brings when it is feasible, it is important that we work to
	expand the fraction of scenarios in which the computational components of research can be automatically 
	repeated exactly over ranges of time and space relevant to scientific research and discourse.
These efforts are particularly important for the research community to pursue, and for science funding
	agencies to support, because the computing industry generally does not have requirements for exact 
	repeatability across significant spans of time.

Again---it is important to acknowledge that the concept of exact repeatability is 
	qualitatively different from the concept of reproducibility that underlies the natural sciences.
In particular, scientific reproducibility is not simply a weaker form of computational repeatability.  
\emph{Approximating or achieving computational repeatability does not automatically deliver scientific reproducibility.}

It is in a sense both bad and good news that exact computational repeatability is not tantamount to scientific reproducibility.
The disappointing news, perhaps, is that it is possible to put much effort into achieving computational repeatability,
	exact where practical and inexact otherwise,
	without delivering the kind of reproducibility that is critical for producing trustworthy science.
The good news is that scientifically meaningful reproducibility can be realized in cases (or over spans of time)
	where computational repeatability is impractical due to the limitations of available technology or affordable resources.
Thus, the older concept of reproducibility that permeates the basic natural sciences has a very
	useful role even where digital computing makes exact repeatability a theoretical possibility.

 Researchers in the natural sciences are comfortable with the idea that it is not possible to exactly
	repeat all reported observations, procedures, and experimental results.
They do not see this as a contradiction to their demand that science be reproducible.
What the natural sciences actually do demand is that 
	(a) research procedures be repeatable by others in principle;
	(b) the means of repeating the work be subject to review and evaluation; 
	and (c) such review and evaluation be possible \emph{without} actually repeating the work.
To be perfectly clear about the third demand: in the natural sciences it is actually considered a 
	\emph{problem} if exact repetition of the steps taken in reported research is required either
	to evaluate the work or to reproduce results.

Consequently, it is not necessary to achieve or 
	maintain perfect repeatability of the computational components of research for scientists to 
	consider a study reproducible and therefore trustworthy.
At the same time it is important that the standards, technologies, 
	computational best-practices, and infrastructure we develop and advocate in fact support scientific reproducibility.
It is not enough, in the long run, to pursue and support exact computational repeatability where we can, 
	and to get as close as possible otherwise.
Rather, computational repeatability is best seen as a dimension of research reproducibility \emph{orthogonal} to 
	the dimension of transparency.
It is possible to achieve computational repeatability without providing research transparency---and vice versa.
Moreover, exact repeatability is not an essential element of scientific reproducibility in the broadest sense of the term.
Transparency arguably is.

\section{Terminology}

What are some specific ways that Research Objects can help make scientific research more transparent?
Many of the objectives and current capabilities of Research Objects already can be seen as supporting 
	transparency.
Examples of Research Objects support research reproducibility by enhancing transparency include...

In the remainder of this paper we propose that Research Objects can help in additional ways that not
	just enhance the transparency of research, but also ensure that transparency and other key elements
	of scientific reproducibility can be achieved, described, and shared meaningfully for all domains
	of research---including those that include both an experimental and computational elements.

The first way in which Research Objects can help is by helping researchers safely navigate the 
	terminological quagmire surrounding the definitions of terms such as \emph{reproducible},
	\emph{replicable}, and \emph{transparency}.
A very simple yet important use case for Research Objects could be the declaration of the senses in
	which the research study and results associated with the Object are in fact reproducible, replicable,
	computationally repeatable, and so on.
Before extending or depending on others' works, methods, or results in their own studies, researchers
	reasonably want to know if that previous work is reproducible in various senses of the word.
Research Objects can help, not just be providing a place to make such declarations, but by preventing
	misunderstandings of what is meant by particular terms.

The current debate over the meaning of key terms describing 
	scientific reproducibility are motivated primarily by a desire to avoid just such confusion.
The recommendations from the Federation of
	American Societies for Experimental Biology (FASEB) cite "lack of uniform definitions to describe the problem" 
	as one of the top three factors that "impede the ability to reproduce experimental results."
 The report National Academy of Sciences Committee on Reproducibility and Replicability of Science states
	that "the difficulties in assessing reproducibility and replicability are complicated by this absence of
	standard definitions for these terms."

The recommendations of these two organizations are representatives of numerous recent studies, papers, 
	and proposed definitions intended to enhance reproducibility by providing a uniform terminology
	for describing it.  
The FASEB recommendations originate in one domain of science while the NAS definitions explicitly 
	"are intended to apply across all fields of science."
Given the interdisciplinary character of modern research---and in particular the ubiquity of computing in science---it 
	is hard to argue against attempts to facilitate communication about reproducibility across science as a whole.

What can be surprising to researchers new to this debate is how many ways the proposed definitions
	can differ.
First, there is disagreement over which term, \emph{reproducibility} or \emph{replicability}, indicates
	 a greater adherence to the procedures, material,  and methods employed in the original research.
The FASEB definitions (in accordance with the terminology around \emph{replicates} described in section 4)
	require from \emph{replicability} a greater fidelity to the original study:

	Replicability: the ability to duplicate (i.e., repeat) a prior result using the same
	source materials and methodologies. This term should only be used when
	referring to repeating the results of a specific experiment rather than an
	entire study.

	Reproducibility: the ability to achieve similar or nearly identical results using comparable materials and methodologies. 
	This term may be used when specific findings from a study are obtained by an independent group of researchers

According to FASEB, \emph{replicability} indicates a higher degree of fidelity than does \emph{reproducibility}, 
	both with respect to the prior result to be confirmed, and to the materials and methodologies employed.
Replicability also appears likely more feasible for the original researchers (they presumably have access to the 
	"same source materials" and are in the best position to use the same methodologies), whereas reproducibility is 
	feasible for "an independent group of researchers". 
Both definitions may be applied to experimental results, but neither definition precludes application to in silico 
	experiments or to the computational elements of laboratory studies.

 In contrast, the definitions in the recent report from the National Academy of Sciences reverses the relative fidelity
implied by the  terms 'reproducibility' and 'replicability':

	Reproducibility is obtaining consistent results using the same input data, computational
	steps, methods, and code, and conditions of analysis. 

	Replicability is obtaining consistent results across studies aimed at answering the same
	scientific question, each of which has obtained its own data.

The NAS definition of replicability is most similar to the FASEB definition of reproducibility.
This reversal of the meanings of these terms between various research domains is well documented.
This aspect of the disagreement over terminologies is in a sense trivial, although the NAS rightly 
	states that the "different meanings and uses across science and engineering" has "led to confusion in collectively 
	understanding problems in reproducibility and replicability."
It is interesting that the NAS report does not suggest new terms for referring to the \emph{technical replicates} 
	and  \emph{biological replicates} so important in experimental biology, should biologists adopt the recommendation 
	of restricting \emph{replication} to "obtaining consistent results across studies".
FOOTNOTE: The NAS report section "Precision of Measurement" quotes a portion of the International Vocabulary of
	Metrology that twice employs the term \emph{replicate measurement}.
What might come as news to biologists is the assertion that the NAS "committee adopted specific definitions" of 
	reproducibility and replicability, "which are otherwise interchangeable in everyday discourse."
Not only is the high-fidelity \emph{replication} of DNA (in the \emph{replisome}) and the lower fidelity \emph{reproduction}
	of organisms  matters for everyday discourse for the many biologists study these processes in nature or employ them in the lab,
	it is easy to see an analogy between replication of DNA and careful replication of measurements and samples
	in the lab on the one hand, and on the other the reproduction of organisms where variation is encouraged in nature
	(for example through sex) and the reproduction of scientific results across studies where, again, some variation is both 
	expected and desirable.

A far more intriguing aspect of the NAS definitions is that experiments not carried out entirely in silico apparently are 
	left with only the term \emph{replicability}.
Satisfying the definition of reproducibility requires "computational steps" and "code", and the report goes on to clarify 
	that reproducibility "is synonymous with computational reproducibility,"  and "the terms are used interchangeably in this report."
Indeed the executive summary of the report states not only that "We define reproducibility to mean computational reproducibility" 
	but also that "the committee adopted definitions that are intended to apply across all fields of science."
The clear implication is the term \emph{reproducible} only can be applied only to the computational components of research.
Because this term is analogous to \emph{replicable} as defined by FASEB, the NAS definitions do not provide a vocabulary 
	that would enable experimentalists to report the intrinsic repeatability of their own methods, measurements, and results. 

Intriguing similarities and differences also appear in definitions of transparency. According to FASEB, transparency is

	The reporting of experimental materials and methods in a manner that provides enough information 
	for others to independently assess and/or reproduce experimental findings

While the NAS report states:

	When a researcher transparently reports a study and makes available the underlying digital artifacts, such as data and code, 
	the results should be computationally reproducible.

According to the latter definition, transparency, like reproducibility, requires digital artifacts which could be of concern to
	those expecting experimental procedures to be transparent.
However both definitions imply that transparency is a necessary \emph{component} of reproducibility, a position that
	suggests a role for Research Objects to play in the resolving this terminological conundrum.

In short, we propose that users of Research Objects be provided with a vocabulary for asserting and querying the reproducibility
	of studies, results, and methods along multiple dimensions. 
Namespaces would support multiple definitions of terms without conflict.
Synonym relationships and other mappings between the vocabularies would enable reasoning about reproducibility
	 and support assertions and queries phrased using terminologies selected by the user.
For example, a researcher publishing a research object might assert that the study is reproducible \emph{sensu} Whole Tale.
Another researcher filtering discovered Research Objects by the property NAS::reproducible would find this study
	either if WT::reproducible had been found to imply NAS::reproducible generally, or if other assertions made by the author
	about the Object satisfy the requirements of the latter term in conjunction with the implications of  WT::reproducible.

Going forward, the Whole Tale project aims to explore the various terminologies for reproducibility with the goal of identifying 
	what might be considered to be the principle components of reproducibility in science as a whole.
As a community we could then determine how various terms and definitions can be seen as compositions of these components.
This in turn would reveal how Research Object infrastructure should reason about these terms, and  how claims 
	made in terms of one set of definitions could be converted to claims using another set of definitions.



\section{Computational Challenges}

		The fundamental limitations computers impose on replicability of program executions are well known.

			Finite precision arithmetic, different word sizes on different processors, round-off errors, etc, impose limits on scientific computations and
				their replicability across different computing environments.
			Virtual machines and containers do not address these issues. Full emulation is required to run the same binary in identical fashion
				on a different processor.  This is typically slow.
			These limitations are even more challenging to manage reproducibly because programs typically are compiled, meaning that the
				exact sequence of machine instructions executed even by a single processor cannot generally be controlled.  A different compiler,
				or a newer version of the same compiler will yield a different sequence of machine instructions.

		Replicating the outputs of a program is far from straightforward

			Observing that a program or set of programs can be executed again is not sufficient to conclude that the underlying computation
				was replicated.  The outputs of the programs must be checked for equivalence.
			Because of the expected variation in run time behavior of programs due to the issues above, checking that outputs of a program
				run are equivalent to the outputs of a run of that program is not always as simple as
				comparing the outputs for bitwise identity.
			Robustly checking for equivalence of output generally must be confirmed in some way other than comparing files at the bit level.
			Footnote: The excellent practice of including accurate provenance and other meaningful metadata in data file headers makes it
				even more unlikley that outputs from different runs will be bitwise-identical.

		Replicating just the software running the program is challenging in practice

			How can we ensure that the stream of instructions sent to the processor for two executions is identical?
			Even holding the computer hardware and compiler version constant, programs depend on language libraries, OS libraries, and system calls.
			Much scientific software also depends directly and indirectly on large numbers of 3rd-party libraries.
			Only direct dependencies can be controlled reliably at build-time.  And many dependencies are via shared libraries that can change between executions
				of the exact same executable--no recompile is needed to get a new effective executable.
			Footnote:  Fans of the Go programming language are bringing back the static executable.
			Recompiling or even just rerunning the "same" program a week later can result in a completely different effective instruction stream.

		Even reproducing computing environments is hard

			Containers and their discontents
				Footnote:  By 'discontents' we do not mean that we object to the use of containers, but rather than we are not content with container technology alone.
				There currently is much enthusiasm around containers as a means of reproducing computing environments and making computational science replicable.
				Whole Tale is one of several projects leveraging the capabilities of containers for this purpose.  Others include Binder and Code Ocean.
				In Whole Tale it is recognized that containers alone cannot satisfy researcher's needs for sharing their computing environments and computations.
				Rather, container technology such as Docker provide an invaluable tool for the reproducible science software stack architect.
				A major motivation for funding (and continuing to fund) projects like Whole Tale is that the containers on their own are insufficient as means
					to making computational science reproducible, and it is not practical for individual researchers and groups to use containers
					and other technology to actually achieve scientifically meaningful reproducibility over periods longer than the publication-cycle time scale.

			What containers do not do
				Despite what sound like suggestions to the contrary in the literature, container technology such as Docker do not ensure computational replicability,
					and do not on their own solve any of the problems of computational replicability described above.
				What containers do provide a very convenient means for executing customized computing environments on behalf of researchers, withot having to run
					an entire virtual machine for each environment.
				In common with virtual machine technology, containers do not abstract or hide the underlying hardware architecture of the computer on which they run.
				They do not abstract the underlying operating system, but simply use the Linux kernel on the host.  Kernel parameter settings on the host apply to
					all containers running on the host (reference famous blog post on the topic "Containers Do Not Contain").
				Rebuilding an image from its Dockerfile specification is not guaranteed to yield the same image.  It general it will not.
				Container images, once they are built, are not guaranteed to run on future releases of the container host.
				They also do not ensure that computations run within the container will replicable in the feature.

			What containers are for
				What containers are good for is precisely what they were to do for the computing industry:  enable developers to write and test code in a
					computing environment of a developer's choosing that can then be replicated on a very short time scale (hours or days) in staging
					and production environments.
				Containers also are good at managing conflicts in dependencies between different components of a multiprocess software architecture.
				Using containers to 'contain' dependencies in this way is most effective when an application can be split across multiple containers running in concert.
				The model of one container, one computing environment does not lend itself to dependency isolation.

			The problem of time and dependencies
				An emerging threat to reproducibility of computational science is the spread misconception that sharing the definition of a container image,
					e.g. by including a Dockerfile in the Git repo for the project, is a guarantee that others (or even the original researcher) will be able to
					recreate the corresponding image and computing environment it represents.
				Researchers making this assumption may be less likely to preserve all of the information actually required to reproduce their computations.
				A major reason a Dockerfile is not enough is that the implicit dependencies of the built environment are constantly changing.
				This is well known to anyone working directly with Docker, or other container technologies.
				Here we will give a single example of the implications of this issue for reproducible science.

	What is reproducibility really for?

		Achieving meaningful replicability even for the computational parts of research is very challenging.
		But this is no reason to give up hope.
		Replicability is a means to an end--justification of scientific results--and Research Objects can help us achieve that end by other means.
		What is most exciting about Research Objects is that they can achieve this end despite the difficulty of computational replicability.
		And the primary means by which Research Objects can due this is by providing transparency.




%\section{Example Provenance Questions}

%Let's do some \emph{italics} and \texttt{teletype} and \textsf{sans
%  serif}. Does it work?
%Here's a bit of code, very simple via \verb|some verbatim| in line, or
%\begin{verbatim}
%.. in some separate verbatim paragraphs ..
%Does it work?
%   Even indentation should work .. 
%\end{verbatim}


%\section{Future Work}


% \begin{acks}
% Work supported in part by NSF award 1541450 (Whole Tale).
% \end{acks}

%
% The next two lines define the bibliography style to be used, and the bibliography file.
%\bibliographystyle{alpha-initials-big}
\bibliographystyle{abbrv}
\bibliography{main}

\end{document}
