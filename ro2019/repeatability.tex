\section{Computational Challenges}

		The fundamental limitations computers impose on replicability of program executions are well known.

			Finite precision arithmetic, different word sizes on different processors, round-off errors, etc, impose limits on scientific computations and
				their replicability across different computing environments.
			Virtual machines and containers do not address these issues. Full emulation is required to run the same binary in identical fashion
				on a different processor.  This is typically slow.
			These limitations are even more challenging to manage reproducibly because programs typically are compiled, meaning that the
				exact sequence of machine instructions executed even by a single processor cannot generally be controlled.  A different compiler,
				or a newer version of the same compiler will yield a different sequence of machine instructions.

		Replicating the outputs of a program is far from straightforward

			Observing that a program or set of programs can be executed again is not sufficient to conclude that the underlying computation
				was replicated.  The outputs of the programs must be checked for equivalence.
			Because of the expected variation in run time behavior of programs due to the issues above, checking that outputs of a program
				run are equivalent to the outputs of a run of that program is not always as simple as
				comparing the outputs for bitwise identity.
			Robustly checking for equivalence of output generally must be confirmed in some way other than comparing files at the bit level.
			Footnote: The excellent practice of including accurate provenance and other meaningful metadata in data file headers makes it
				even more unlikley that outputs from different runs will be bitwise-identical.

		Replicating just the software running the program is challenging in practice

			How can we ensure that the stream of instructions sent to the processor for two executions is identical?
			Even holding the computer hardware and compiler version constant, programs depend on language libraries, OS libraries, and system calls.
			Much scientific software also depends directly and indirectly on large numbers of 3rd-party libraries.
			Only direct dependencies can be controlled reliably at build-time.  And many dependencies are via shared libraries that can change between executions
				of the exact same executable--no recompile is needed to get a new effective executable.
			Footnote:  Fans of the Go programming language are bringing back the static executable.
			Recompiling or even just rerunning the "same" program a week later can result in a completely different effective instruction stream.

		Even reproducing computing environments is hard

			Containers and their discontents
				Footnote:  By 'discontents' we do not mean that we object to the use of containers, but rather than we are not content with container technology alone.
				There currently is much enthusiasm around containers as a means of reproducing computing environments and making computational science replicable.
				Whole Tale is one of several projects leveraging the capabilities of containers for this purpose.  Others include Binder and Code Ocean.
				In Whole Tale it is recognized that containers alone cannot satisfy researcher's needs for sharing their computing environments and computations.
				Rather, container technology such as Docker provide an invaluable tool for the reproducible science software stack architect.
				A major motivation for funding (and continuing to fund) projects like Whole Tale is that the containers on their own are insufficient as means
					to making computational science reproducible, and it is not practical for individual researchers and groups to use containers
					and other technology to actually achieve scientifically meaningful reproducibility over periods longer than the publication-cycle time scale.

			What containers do not do
				Despite what sound like suggestions to the contrary in the literature, container technology such as Docker do not ensure computational replicability,
					and do not on their own solve any of the problems of computational replicability described above.
				What containers do provide a very convenient means for executing customized computing environments on behalf of researchers, withot having to run
					an entire virtual machine for each environment.
				In common with virtual machine technology, containers do not abstract or hide the underlying hardware architecture of the computer on which they run.
				They do not abstract the underlying operating system, but simply use the Linux kernel on the host.  Kernel parameter settings on the host apply to
					all containers running on the host (reference famous blog post on the topic "Containers Do Not Contain").
				Rebuilding an image from its Dockerfile specification is not guaranteed to yield the same image.  It general it will not.
				Container images, once they are built, are not guaranteed to run on future releases of the container host.
				They also do not ensure that computations run within the container will replicable in the feature.

			What containers are for
				What containers are good for is precisely what they were to do for the computing industry:  enable developers to write and test code in a
					computing environment of a developer's choosing that can then be replicated on a very short time scale (hours or days) in staging
					and production environments.
				Containers also are good at managing conflicts in dependencies between different components of a multiprocess software architecture.
				Using containers to 'contain' dependencies in this way is most effective when an application can be split across multiple containers running in concert.
				The model of one container, one computing environment does not lend itself to dependency isolation.

			The problem of time and dependencies
				An emerging threat to reproducibility of computational science is the spread misconception that sharing the definition of a container image,
					e.g. by including a Dockerfile in the Git repo for the project, is a guarantee that others (or even the original researcher) will be able to
					recreate the corresponding image and computing environment it represents.
				Researchers making this assumption may be less likely to preserve all of the information actually required to reproduce their computations.
				A major reason a Dockerfile is not enough is that the implicit dependencies of the built environment are constantly changing.
				This is well known to anyone working directly with Docker, or other container technologies.
				Here we will give a single example of the implications of this issue for reproducible science.

	What is reproducibility really for?

		Achieving meaningful replicability even for the computational parts of research is very challenging.
		But this is no reason to give up hope.
		Replicability is a means to an end--justification of scientific results--and Research Objects can help us achieve that end by other means.
		What is most exciting about Research Objects is that they can achieve this end despite the difficulty of computational replicability.
		And the primary means by which Research Objects can due this is by providing transparency.
