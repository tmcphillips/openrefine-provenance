\section{Transparency}

In this paper we argue that the dimension of reproducibility most ripe for the contributions of computer science
	is research \emph{transparency}, in particular through the modeling, recording, and querying of the provenance of research artifacts.
In alignment with researchers in the natural sciences who recognize transparency as crucial,
	we are confident that provenance management has much to contribute to scientific reproducibility,
	 even when it does not specifically enable exact repeatability of the computations they describe.

For provenance management systems, representations, and user interfaces to support reproducibility via transparency,
	however, they must support science-oriented queries.
Provenance must be able to answer questions about the \emph{science} that was performed--not just the
	sequence, dependencies, and flow of data through computational steps.
The answers to these questions must enable others to evaluate the scientific quality of the work, and to learn what is necessary to
	reproduce the results without actually repeating every step taken in the original work.
Provenance must enable researchers to build on the results and processes reported in prior work with confidence.

Finally, it must be possible for researchers unversed in the detailed specifications of Research Objects and the PROV standard
	to pose questions and receive answers meaningful for evaluating, using, and building on the
	processes and products of prior research.
We suggest that Research Objects and related approaches are the ideal vehicle for storing, sharing and making
	provenance queryable in this way.
Research Objects thus can support scientific reproducibility even in the face of the many practical challenges to
	computational repeatability.
