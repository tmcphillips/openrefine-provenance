\section{Reproducibility in science}\label{sec-reproducibility}

Modern science is founded on the expectation that the observations, experiments, and
	predictions that comprise scientific research be independently verifiable by others.
This requirement, referred to as the \emph{reproducibility} or \emph{replicability} of 
	science, applies not only to the products of research studies (\emph{substances}, 
	\emph{results}, \emph{conclusions}, \emph{models}, \emph{data products}, 
	\emph{predictions}), but also to the activities that ultimately give rise to these
	products (\emph{methods}, \emph{protocols}, \emph{workflows}); the materials 
	employed in these activities (\emph{reagents}, \emph{instruments}, 
	\emph{software}); and the conditions 
	under which which these activities are carried out (\emph{temperatures},
	\emph{instrument settings}, \emph{software parameters},
	\emph{computing environments}).  When sufficient details are available
	such that the research products and methods can be reviewed, interpreted, and
	evaluated by other researchers \emph{without} repeating the work, a study is said to be 
	\emph{transparent}.

While it is true that studies attempting primarily to reproduce previous results are relatively rare in the
	pure natural sciences, even the most groundbreaking studies in these fields include components	
	that explicitly or implicitly confirm the reproducibility of previously reported results and procedures.
The expectation is that new studies will reliably produce meaningful results consistent with previous work 
	only if the prior work on which they are based or otherwise relates to is reproducible.
In this sense, the whole of basic research in the natural sciences can be seen as an ongoing, massively-parallel
	reproducibility study that also happens to produce a steady stream of new results.
Exceptions to this pattern occur when studies appear to overturn
well-established understandings of nature \cite{kuhn1962structure},
	violate the expectations of how research in a particular field is to be carried out, or otherwise cause controversy.
In these cases direct attempts may be made to reproduce results by duplicating as carefully as possible
	the reported methods and conditions described in the controversial study.

Even when attempts are made specifically to confirm the reproducibility of particular studies or results, investigators in
	the natural sciences generally do not expect the processes and products of research to be duplicated exactly.
The vast majority of quantitative observations made of real world phenomena using scientific instruments
	are associated with limited precision and other intrinsic uncertainties that must themselves be characterized
	and well understood for science based on them to be considered reproducible.
It is a hallmark of trustworthy science that quantitative observations and claims are inseparable from these 
	uncertainties in measurement and their propagation through data analysis.

Similarly, the materials and processes employed in the natural sciences generally are impossible
	to duplicate exactly.
In a chemistry laboratory, the precise quantities of input reagents will vary, temperatures will differ, and heating
	or cooling rates will be unique for each run of a chemical synthesis, no matter how carefully these conditions
	are controlled; the yield and purity of the intended product necessarily will vary as well from run to run.
A similar situation holds when measurements are made on samples using a scientific instrument. 
Different instruments of the same model will vary slightly and produce slightly different results even
	under identical conditions on identical samples.
Generally, the original researchers are in the best position to assess how the minimum variation expected
	between runs of a synthesis (they have access to the same batch of reagents and the same equipment), 
	or between repeated readings of an instrument on the same or equivalent samples (they can prepare
	multiple samples at the same time, and run these samples through the instrument one after the other).
A researcher attempting to duplicate another's work can expect to see greater deviation from the reported results	
	because the materials and conditions involved will necessarily differ to a greater degree.

This asymmetry between the original researcher and another repeating the work is reflected in the longstanding
	distinction between \emph{reproducibility} and \emph{replicability} in experimental biology.  
In  Section\,\ref{sec-repeatability}  we will examine definitions of these terms jointly adopted by twenty-nine research
	societies in the biological sciences.  
For now we note that the notion of \emph{replicates}, repeated measurements made to quantify 
	experimental variability, is represented by a rich literature.
This literature distinguishes between distinct modes of experimental replication.
The term \emph{technical replicates}, for example, refers to repeated measurements performed on the same sample.
These are used to assess the variation intrinsic to the procedure, apparatus, and instrument employed.
\emph{Biological replicates} represent measurements made on different but equivalent samples.
In practice both generally are performed by the original researcher under conditions otherwise
	held as constant as possible.\footnote{Generating multiple gigabytes of raw data requiring intensive computational analysis for each replicate, Next-Generation Sequencing (NGS) represent just one sub-domain where the reproducibility 
	terminologies in the natural sciences and in computing
        unavoidably collide.}

% BL: not sure where exactly the footnote should be attached ... 


%%% Local Variables: 
%%% mode: latex
%%% TeX-master: "main"
%%% End: 
