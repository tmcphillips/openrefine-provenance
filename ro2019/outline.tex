\section{Outline}
In the remainder of this paper 
we briefly discuss four topics we plan to investigate
	in the course of the Whole Tale project.
In Section\,\ref{sec-reproducibility} we review the general notion of
reproducibility in science, and in Section\,\ref{sec-repeatability} 
	highlight how digital computing in principle makes possible a completely new kind of reproducibility: 
	\emph{exact repeatability}. 
We emphasize that the notion of \emph{transparency}---long a critical element of
	reproducibillity in the pure natural sciences---has a role to play even for those computational components
	of research where exact repeatability is feasible.
In Section\,\ref{sec-terminology} we provide an overview of several dimensions of the terminological debate around reproducibility
	generally, and propose that a pluralistic approach to defining key terms is essential if a general 
	concept of reproducibility is to be shared across disciplines.
In Section\,\ref{sec-limitations} we summarize a number of limitations on exact repeatability in practice, and in Section\,\ref{sec-transparency}
	show how science-oriented provenance queries can mitigate such limitations by maintaining
 	the transparency most essential to reproducibility in science.
Throughout, we highlight the role that Research Objects \cite{bechhofer2013whya}
	can serve in supporting and maintaining reproducibility 
	by encapsulating the information needed to rerun the computational steps in a study, 
	by disambiguating claims about reproducibility, 
	and by enabling transparency via queries of provenance information packaged in the object. 
 

%%% Local Variables: 
%%% mode: latex
%%% TeX-master: "main"
%%% End: 
