
\section{Introduction}

Publicly funded efforts currently are underway around the world to 
	ensure that the computational components of scientific research
	can be made "reproducible" or "replicable".
The ongoing discourse about the perceived "reproducibility crisis in science"
	is just one illustration of the importance of these efforts.
The energy invested in the wide-ranging debate over the precise meanings of the 
	terms \emph{reproducible}, \emph{replicable}, \emph{transparent}, etc, with 
	respect to research results, processes, and settings is perhaps an even greater indication of 
	both the signficance of these efforts and the challenges they face.
For while each effort aimed at facilitating reproducible computing in the
	sciences must clearly define its mission and apply the bulk of its resources
	to the specific problems it sets out to address, these efforts necessarily do
	so within the context of broader discussions about the nature, importance,
	and precise definitions of the qualities of science we wish to extend to computing
	over the longer time scale.

Within a particular effort it is useful to define terms such as \emph{reproducible} operationally.
For example, in the Whole Tale project we define a \emph{Reproducible Tale} as one 
	that \emph{includes sufficient information for the Tale to be re-executed for the review 
	and verification of results}.
Adopting this definition allow us to focus our requirement analysis, system design,
	and software implementation efforts on the specific problems Whole Tale is funded to solve
	and use cases we aim to support.
Supporting publishers who require submissions to include all new data, 
	code, and workflows needed to reproduce all computed artifacts supporting
	claims in a paper is one such use case we are targeting in Whole Tale.
We anticipate that facilitating the re-execution of the code used to produce
	the key products of a study will enable publishers to routinely confirm that
	the provided data and code do in fact produce those results, thus addressing
	a key dimension of the reproducibility challenges currently facing science.

At the same time, it is critical that efforts like Whole Tale contribute to a global 
	vision of computational reproducibility in the sciences, and clearly situate 
	its particular mission, use cases, and engineering deliverables in this context.
For while the particular technical problems that Whole Tale and similar projects currently
	aim to address are particularly pressing, their efforts by no means
	represent the entire landscape of concepts, problems, and technical solutions
	under discussion.
In particular, current engineering efforts are unlikely to elevate the computational components
	of research to the level of reproducibility expected of studies in the
	pure natural sciences such as physics, chemistry and biology.

Consequently, we view the Whole Tale project--as currently chartered and funded--as just a step
	towards the kind of platforms, infrastructure, and standards
	needed to enable researchers using computing technology to routinely 
	achieve the reproducibility long considered the essence of science as a whole.
In support of this longer-term vision, we outline in this paper just a few of the issues
	we aim to discuss with the broader community in the medium term.
We anticipate that development of later iterations of Whole Tale and its sibling efforts 
	will be driven in part by the problem definitions and solution proposals we collectively
	develop between now and then.

 
