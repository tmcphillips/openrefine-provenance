
\section{Introduction}

Efforts currently are underway in publicly funded projects around the world to 
	ensure that the computational components of scientific research
	can be made "reproducible" or "replicable".
The ongoing discourse about the perceived "reproducibility crisis in science"
	is just one illustration of the importance of these efforts.
The energy invested in the wide-ranging debate over the precise meanings of the 
	terms reproducible, replicable, transparent, etc, with respect to research
	results, processes, and settings is perhaps an even greater indication of 
	both the signficance of these efforts and the challenges that they face.
For while each effort aimed at facilitating reproducible computing in the
	sciences must clearly define its mission and apply the bulk of its resources
	to the specific problems it sets out to address, these efforts necessarily do
	so within the context of broader discussions about the nature, importance,
	and precise definitions of the qualities of science we wish to bring to computing
	over the longer time scale.

Within a particular effort it is useful to define terms such 'reproducible' operationally.
For example, in the Whole Tale project we define a \emph{reproducible tale} as one 
	that includes sufficient information for the Tale to be re-executed for the review 
	and verification of results.
Adopting this definition allow us to focus our requirements analysis, design,
	and implementation efforts on the specific problems Whole Tale is funded to solve,
	and the overarching use cases we aim to support.
Supporting publishers who would like to require submissions to include all new data, 
	code, and workflows required to reproduce all data artifacts used to support
	claims in the paper is one such use case we are targeting in Whole Tale.

At the same time, it is critical that efforts like Whole Tale contribute to a globally 
	shared vision of reproducibility in the sciences generally, and clearly situate 
	its particular mission, use cases, and engineering deliverables in this context.
For while the particular technical problems that Whole Tale and similar projects currently
	aim to address are both particularly pressing at the moment, their efforts by no means
	represent the entire landscape of concepts, problems, and technical solutions
	under discussion.
In particular, the foci of current efforts are unlikely to elevate the reproducibility
	of scientific research with significant computational elements to the
	level of reproducibility considered not just to be essential to, but also the essence of,
	the pure natural sciences such as physics, chemistry and biology.

Consequently, we view Whole Tale as currently chartered and funded as just a step
	towards the kind of platforms, infrastructure, standards, and research environments
	we see as needed to enable researchers to use computing technology in research
	while reliably providing the kind of reproducible long assumed as critical for
	science as a whole.
In support of this longer-term vision we describe in this paper some of the issues
	that we see as critical to discuss as a community now, with the expectation that
	development of later iterations of Whole Tale and its sibling efforts 
	infrastructure and tools will be driven in part by the problem definitions and
	solution proposals we collectively develop between now and then.



 
