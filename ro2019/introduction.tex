\section{Introduction}

The practice of modern science is founded on the expectation that the observations, experiments, and
	predictions that comprise scientific research can be independently verified by others.
This requirement is referred to either as the \emph{reproducibility} or \emph{replicability} of 
	science.
These terms apply not only to the products of research studies (\emph{results}, 
	\emph{conclusions}, \emph{models}, \emph{data products}, \emph{predictions}), but also to the 
	essential activities that take place during research (\emph{methods}, \emph{protocols}, \emph{workflows}) 
	and that ultimately give rise to these products; the materials employed in and necessary to these 
	research actitivties (\emph{reagents}, \emph{instruments}, \emph{software}); and to the conditions under which 
	which these activities are carried out (\emph{instrument settings}, \emph{software parameters},
	\emph{computing environments}).  When sufficient details are available
	such that the research products and methods can be reviewed, interpreted, and
	evaluated by other researchers \emph{without} repeating the work, a study can be 
	considered \emph{transparent}.

Although reports of work reproducing the results of other researchers are relatively rare in the natural
	sciences, new studies in these fields nearly always contains elements that explicitly or implicitly 
	confirm the reproducibility of previous results and reported procedures.
The general expectation is that new studies employing previously described methods or building 
	on published research results will themselves produce meaninful results only if the prior work 
	on which they depend is reproducible.
Conversely, it is is when new research unexpectedly fails to produce meaningful results, or else contradicts
	prior results, that the reproducibility of prior work is called in to question.
In this sense much of basic research in the natural sciences can be seen as a  
	a massively-parallel reproducibility study that also happens to produce new results.
The exception when reported results appear to overturn well-established understandings of nature,
	appear to violate the expectations of how research in a particular field is to be carried out,
	or is otherwise controversial.
In these cases direct attempt may be made to reproduce results by precisely replicating the reported methods
	and conditions reported in the controversial study.
	
Digital computing approaches make it possible to repeat exactly certain computational aspects of research
	to an extent that exceeds what can be achieved when observing or experimenting with natural 
	phenomena in the physical world.  
It generally is expected that computational processes, the implementation of hardware and software 
	enabling those processes, and the outputs of those processes all can be repeated exactly--at least in principle.
Where computing makes up a significant fraction of scientific research, the terms reproducibility and
	replicability have come to associated with this radically different expectation of exact repeatability. 
Moreover, while this expectation of exact repeatability may be realizable in principle, in practice the complexities of real-world
	hardware and software make  computational repeatedability very challenging to achieve in practice except in very limited cases.
Currently much effort is going into expanding the fraction of scenarios in which the computational
	components of research can be automatically repeated exactly over ranges of time and space relevant to scientific 
	research and discourse.
These efforts are important for the research community to pursue, and for science funding
	agencies to support, especially because the computing industry generally does not have 
	requirements for exact repeatability across signficant spans of time.
However, it also is crucial to note that the concept of exact repeatability of the kind pursued by these efforts is 
	qualitatively different from the concept of reproducibility (or replicability) that underlies the natural sciences.
It is even more important to realize that scientific reproducibility is not simply a weaker form of computational
	repeatability.  
The relationship between repeatability and reproducibility as the terms are used here is not one of degree.
In particular, achieving computational replicability does not automatically deliver scientific reproducibility.
This is both bad and good news.
The bad news is that it is possible to put much effort into achieving computational repeatability without delivering
	the kind of reproducibility that is critical for producing trustworthy science.
The good news is that scientifically-meaningful reproducibility can be realized in cases (or over spans of time)
	where computational repeatability is either impossible or impractical given readily available technology.
Reseachers in the natural sciences are comfortable with the idea that it may not be possible to exactly
	repeat all reported observations, procedures, and experimental results.
They do not see this concession to real-world practicalities as a contradiction to their demand that science be reproducible.
What the natural sciences actually do demand is that:
	(a) research procedures be repeatable by others in principle
	(b) the means of repeating the work be subject to review and evaluation
	(c) such review and evaluation be possible without actually repeating the work
	(d) it not be required to repeat the exacts taken in reported research to reproduce the results
In other words, in the natural sciences it is actually considered a problem if exact repetition of the steps taken in reported research
	is required either to evaluate the work or to reproduce results.
The reason this is good news for computation-intensive research is that it is not necessary to achieve or maintain perfect repeatability
	of the computational components of research in order for scientists to consider the work reproducible and therefore trustworthy.
However, this also means that the standards, technologies, and computational best-practices that we develop and advocate
	in fact support scientific reproducibility. Pursuing and supporting exact computational repeatedly is not enough.
In this paper we argue that the necessary ingredient of reproducible research most at risk of left out when supporting computational
	repeatability is transparency.
We emphasisze that an area in which computer science has much to offer in supporting transparency is in the modeling, recording, 
	and querying of the provenance of research artifacts.
But we also point out that for provenance to support reprodudibility it must support science-oriented queries.
Provenance must be able to answer questions about the science that was performed--not just the computations.
Answers to these questions must enable others to evaluate the scientific quality of the work, and to learn what is necessary to 
	reproduce the results without blindly repeating every step taken in the original work.
And we suggest that Research Objects and related approaches are the ideal vehicle for storing and make provenance queryable
	in this way, thus supporting true scientific reproducibility.