\section{Introduction}

The challenge of reproducibility
	Challenge of definition
		Terminology debate: Reproduce vs Replicate
			The difference between them
			Standardization proposals
			History of usage in biology
				Cell and molecular biologists study both replication and reproduction as natural processes.
				DNA replicates (high fidelity, variation not desired, ingredients indistiguishable).
				Organisms reproduce (lower fidelity, variation desired, different ingredients).
				Cells have replisomes, complex molecular machines where DNA replication occurs, and copying errors are detected and corrected.
				In origin of life research a crucial debate is over 'replication first' (DNA World) or 
 					'metabolism first' (aka reproduction first, i.e. without replicating genetic material).
				These terminologies are well established.
				Biologists also have a rich and well-defined vocabulary to describe replicability of experimental measurements and results.
				Commonly distinguish two kinds of experimental 'replicates':  technical replicates, and biological replicates.
				FASEB definitions of reproducibility and replicability are consistent with the above.
				Footnote: provide number of FASEB members, list subdisciplines represented by FASEB.
 				Reality is that the terminology is well established in large branches of science already.
			Need for reproduction/replication to mean different things in different fields
				The relationships of corresponding concepts across fields is one of analogy, not identity.
				Exact repeatability is at least theoretically possible and sometimes practical under realistic assumptions when an experiment 
					is entirely in silico and isolated from the outside world.
				As soon as observation of the real world is involved, exact repeatability often is impossible.
				Neither of the types of replicates in experimental biology are exact, although both are measures of repeatability.
				There often is no way to repeat exactly an experiment that involves scientific instruments, physical samples, or experimental apparatus.
				In contrast, it is not unreasonable to talk about exactly repeating a purely computational experiment, at least by the original researcher,
					on the same hardware, close in time to the original experiment.
				In reality, computational repeatability is not as easy as sometimes assumed other under conditions, but fundamentally this is
					a different situation than when a scientific instrument is involved or observations are made of the external world.
			Our recommendation on terminology
				Respect differences between fields of research and different expectations with regard to reproducibility.
				Use the R* words in ways that make their meanings clear in context.
				Do not be surprised if computational sciences turn out not to be representative of science generally.