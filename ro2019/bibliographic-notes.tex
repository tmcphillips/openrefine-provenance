%%%% Proceedings format for most of ACM conferences (with the exceptions listed below) and all ICPS volumes.
\documentclass[sigconf,screen,nonacm]{acmart}
%\documentclass[sigconf]{acmart}

% * CodeOcean ``guaranteed reproducibility'' 
% * GigaScience magazine article 
% * 1920s (?) quote about ``it's all made up''  


% start of document preamble



%\acmConference[TaPP]{Theory and Practice of Provenance}{June 3, 2019}{Philadelphia, PA}
%\setcopyright{rightsretained}

% defining the \BibTeX command - from Oren Patashnik's original BibTeX documentation.
\def\BibTeX{{\rm B\kern-.05em{\sc i\kern-.025em b}\kern-.08emT\kern-.1667em\lower.7ex\hbox{E}\kern-.125emX}}
 
\usepackage[english]{babel}
\usepackage{enumitem}
\setlist[itemize]{leftmargin=3 mm}

% start of the body of the document body.
\begin{document}

% title
\title{Bibliographic Notes}

%\author{Timothy McPhillips  \qquad Bertram Lud\"ascher}
 
% BL: so let's do some manual hacking: 
% \affiliation{%
%   \institution{School of Information Sciences,  University of Illinois at Urbana-Champaign\\ 
%     \texttt{\{tmcphill,ludaesch\}@illinois.edu} }
% }

% This command processes the author and affiliation and title information and builds the first part of the formatted document.
\maketitle


% The next two lines define the bibliography style to be used, and the bibliography file.

Starting point \cite{rauber16primad}

PRIMAD has been studied and applied in different contexts \cite{ferro2016increasing,chapp2019applicability}


\bibliographystyle{alpha-initials-big}
%\bibliographystyle{abbrv}
\bibliography{repro}

\end{document}

