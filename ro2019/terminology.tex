\section{Terminology}

	Recent years have seen a growing debate over the meanings of the terms reproducible and replicable,
		and relationships between the two.  Should one term imply a greater adherence to the procedures
		taken in the original research?  Should one term be reserved for the activities a researcher performs
		to assess the reliability of their own procedures, experiments, and results; leaving the other term to
		refer to activities centered on assessing the quality of the results of others?  Which term should be used
		to mean what? As alluded to in the introduction, the entry of digitial computing into scientific research
		has further complicated these issues by introducing the possibility of exact repeatability.
	For the purposes of this paper we will continue to reserve the terms reproducibility and replicability to
		qualities demanded of scientific research whether digital computers are involved or not.  We use the
		term repeatability to refer to quality of computations that is desirable so far as computers are involved
		in research, but do not consider repeatability in this sense to be a prerequisite of reproducibility or
		replicability even when computers are used in research.
	For the relative meanings of reproducibility and replicability, we follow here the recommendations of the Federation 
		of American Societies for Experimental Biology, an organization comprising twenty-nine scientific 
		societies and representing over 105,000 practicing researchers:

		Replicability: the ability to duplicate (i.e., repeat) a prior result using the same
		source materials and methodologies. This term should only be used when
		referring to repeating the results of a specific experiment rather than an
		entire study.

		Reproducibility: the ability to achieve similar or nearly identical results using comparable materials and methodologies. 
		This term may be used when specific findings from a study are obtained by an independent group of researchers

	Note that by these definitions repliciating a result implies following the processes reported in the original work more closely ("same 
		source mateirals and methodologies"), than when reproducing a result ("comparable materials and methodologies").  
	Neither definition refers specifically to computed results and both apply to experimental results.  In contrast, the recent report from the National
		Academy of Sciences reverses the relative fidelity implied by the terms 'reproducibility' and 'replicability':

	Reproducibility is obtaining consistent results using the same input data, computational
		steps, methods, and code, and conditions of analysis. 

	Replicability is obtaining consistent results across studies aimed at answering the same
		scientific question, each of which has obtained its own data.


	Perhaps most intriguing about the NAS definitions is that experiments not carried out entirely in silico apparently are left with only term
		to describe them: replicability. Satisfying the definition of reproducibility requires "computational steps" and "code", and  
		the report goes on to clarify that reproducibility "is synonymous with 'computational reproducibility,' and the terms are
		used interchangeably in this report."

	FASEB also defines transparency in a manner consistent with our usage here:

		Transparency:  the reporting of experimental materials and methods in a manner that provides enough information 
		for others to independently assess and/or reproduce experimental findings

	In contrast, the NAS report essentially makes reproducibility a prerequisite of transparency: "When a researcher transparently
		reports a study and makes available the underlying digital artifacts, such as data and code, the results should be 
		computationally reproducible."   

	We take the position that it transparency is essential to science, whereas reproducibility as defined by the NAS is not necessarily possible, as detailed below.

	Footnote or note at end of paper:

			Cell and molecular biologists study both replication and reproduction as natural processes.
			DNA replicates :  high fidelity, variation not desired, ingredients indistiguishable, errors are corrected on the fly.
			Organisms reproduce:  lower fidelity expected, variation desired, different ingredients acceptable.
			Cells have replisomes, complex molecular machines where DNA replication occurs, and copying errors are detected and corrected.
			In origin of life research a crucial debate is over 'replication first' (DNA World) or
 					'metabolism first' (aka reproduction first, i.e. without replicating genetic material).
			These terminologies with biology are well established.
			Biologists also have a rich and well-defined vocabulary to describe replicability of experimental measurements and results.
			Commonly distinguish two kinds of experimental 'replicates':  technical replicates, and biological replicates.
 			Reality is that the terminology is well established in large branches of science already.

		Need for reproduction/replication to mean different things in different fields

			The relationships of corresponding concepts across fields is one of analogy, not identity.
			Exact repeatability is at least theoretically possible and sometimes practical under realistic assumptions when an experiment
				is entirely in silico and isolated from the outside world.
			As soon as observation of the real world is involved, exact repeatability often is impossible.
			Neither of the types of replicates in experimental biology are exact, although both are measures of repeatability.
			There often is no way to repeat exactly an experiment that involves scientific instruments, physical samples, or experimental apparatus.
			In contrast, it is not unreasonable to talk about exactly repeating a purely computational experiment, at least by the original researcher,
				on the same hardware, close in time to the original experiment.
			In reality, computational repeatability is not as easy as sometimes assumed (see below), but fundamentally this is
				a different situation than when a scientific instrument is involved or observations are made of the external world.

		Our approach to terminology

			Respect differences between fields of research and different expectations with regard to reproducibility.
			Do not expect standardization to even be meaningful (never mind politically achievable) across domains.
			Instead, only use the R* words in ways that make their meanings clear in context.
			Do not be surprised if computational sciences turn out not to be representative of science generally.

		Specific implications for our contributions to the Research Objects field

			Take care to define R* words precisely when expressing desiderata, describing features, or making or comparing claims about capabilities.
			Do not expect efforts to achieve computational repeatability alone to enable "reproducible science" generally.

