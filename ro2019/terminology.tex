\section{Terminology}

What are some specific ways that Research Objects can help make scientific research more transparent?
Many of the objectives and current capabilities of Research Objects already can be seen as supporting 
	transparency.
Examples of Research Objects support research reproducibility by enhancing transparency include...

In the remainder of this paper we propose that Research Objects can help in additional ways that not
	just enhance the transparency of research, but also ensure that transparency and other key elements
	of scientific reproducibility can be achieved, described, and shared meaningfully for all domains
	of research---including those that include both an experimental and computational elements.

The first way in which Research Objects can help is by helping researchers safely navigate the 
	terminological quagmire surrounding the definitions of terms such as \emph{reproducible},
	\emph{replicable}, and \emph{transparency}.
A very simple yet important use case for Research Objects could be the declaration of the senses in
	which the research study and results associated with the Object are in fact reproducible, replicable,
	computationally repeatable, and so on.
Before extending or depending on others' works, methods, or results in their own studies, researchers
	reasonably want to know if that previous work is reproducible in various senses of the word.
Research Objects can help, not just be providing a place to make such declarations, but by preventing
	misunderstandings of what is meant by particular terms.

The current debate over the meaning of key terms describing 
	scientific reproducibility are motivated primarily by a desire to avoid just such confusion.
The recommendations from the Federation of
	American Societies for Experimental Biology (FASEB) cite "lack of uniform definitions to describe the problem" 
	as one of the top three factors that "impede the ability to reproduce experimental results."
 The report National Academy of Sciences Committee on Reproducibility and Replicability of Science states
	that "the difficulties in assessing reproducibility and replicability are complicated by this absence of
	standard definitions for these terms."

The recommendations of these two organizations are representatives of numerous recent studies, papers, 
	and proposed definitions intended to enhance reproducibility by providing a uniform terminology
	for describing it.  
The FASEB recommendations originate in one domain of science while the NAS definitions explicitly 
	"are intended to apply across all fields of science."
Given the interdisciplinary character of modern research---and in particular the ubiquity of computing in science---it 
	is hard to argue against attempts to facilitate communication about reproducibility across science as a whole.

What can be surprising to researchers new to this debate is how many ways the proposed definitions
	can differ.
First, there is disagreement over which term, \emph{reproducibility} or \emph{replicability}, indicates
	 a greater adherence to the procedures, material,  and methods employed in the original research.
The FASEB definitions (in accordance with the terminology around \emph{replicates} described in section 4)
	require from \emph{replicability} a greater fidelity to the original study:

	Replicability: the ability to duplicate (i.e., repeat) a prior result using the same
	source materials and methodologies. This term should only be used when
	referring to repeating the results of a specific experiment rather than an
	entire study.

	Reproducibility: the ability to achieve similar or nearly identical results using comparable materials and methodologies. 
	This term may be used when specific findings from a study are obtained by an independent group of researchers

According to FASEB, \emph{replicability} indicates a higher degree of fidelity than does \emph{reproducibility}, 
	both with respect to the prior result to be confirmed, and to the materials and methodologies employed.
Replicability also appears likely more feasible for the original researchers (they presumably have access to the 
	"same source materials" and are in the best position to use the same methodologies), whereas reproducibility is 
	feasible for "an independent group of researchers". 
Both definitions may be applied to experimental results, but neither definition precludes application to in silico 
	experiments or to the computational elements of laboratory studies.

 In contrast, the definitions in the recent report from the National Academy of Sciences reverses the relative fidelity
implied by the  terms 'reproducibility' and 'replicability':

	Reproducibility is obtaining consistent results using the same input data, computational
	steps, methods, and code, and conditions of analysis. 

	Replicability is obtaining consistent results across studies aimed at answering the same
	scientific question, each of which has obtained its own data.

The NAS definition of replicability is most similar to the FASEB definition of reproducibility.
This reversal of the meanings of these terms between various research domains is well documented.
This aspect of the disagreement over terminologies is in a sense trivial, although the NAS rightly 
	states that the "different meanings and uses across science and engineering" has "led to confusion in collectively 
	understanding problems in reproducibility and replicability."
It is interesting that the NAS report does not suggest new terms for referring to the \emph{technical replicates} 
	and  \emph{biological replicates} so important in experimental biology, should biologists adopt the recommendation 
	of restricting \emph{replication} to "obtaining consistent results across studies".
FOOTNOTE: The NAS report section "Precision of Measurement" quotes a portion of the International Vocabulary of
	Metrology that twice employs the term \emph{replicate measurement}.
What might come as news to biologists is the assertion that the NAS "committee adopted specific definitions" of 
	reproducibility and replicability, "which are otherwise interchangeable in everyday discourse."
Not only is the high-fidelity \emph{replication} of DNA (in the \emph{replisome}) and the lower fidelity \emph{reproduction}
	of organisms  matters for everyday discourse for the many biologists study these processes in nature or employ them in the lab,
	it is easy to see an analogy between replication of DNA and careful replication of measurements and samples
	in the lab on the one hand, and on the other the reproduction of organisms where variation is encouraged in nature
	(for example through sex) and the reproduction of scientific results across studies where, again, some variation is both 
	expected and desirable.

A far more intriguing aspect of the NAS definitions is that experiments not carried out entirely in silico apparently are 
	left with only the term \emph{replicability}.
Satisfying the definition of reproducibility requires "computational steps" and "code", and the report goes on to clarify 
	that reproducibility "is synonymous with computational reproducibility,"  and "the terms are used interchangeably in this report."
Indeed the executive summary of the report states not only that "We define reproducibility to mean computational reproducibility" 
	but also that "the committee adopted definitions that are intended to apply across all fields of science."
The clear implication is the term \emph{reproducible} only can be applied only to the computational components of research.
Because this term is analogous to \emph{replicable} as defined by FASEB, the NAS definitions do not provide a vocabulary 
	that would enable experimentalists to report the intrinsic repeatability of their own methods, measurements, and results. 

Intriguing similarities and differences also appear in definitions of transparency. According to FASEB, transparency is

	The reporting of experimental materials and methods in a manner that provides enough information 
	for others to independently assess and/or reproduce experimental findings

While the NAS report states:

	When a researcher transparently reports a study and makes available the underlying digital artifacts, such as data and code, 
	the results should be computationally reproducible.

According to the latter definition, transparency, like reproducibility, requires digital artifacts which could be of concern to
	those expecting experimental procedures to be transparent.
However both definitions imply that transparency is a necessary \emph{component} of reproducibility, a position that
	suggests a role for Research Objects to play in the resolving this terminological conundrum.

In short, we propose that users of Research Objects be provided with a vocabulary for asserting and querying the reproducibility
	of studies, results, and methods along multiple dimensions. 
Namespaces would support multiple definitions of terms without conflict.
Synonym relationships and other mappings between the vocabularies would enable reasoning about reproducibility
	 and support assertions and queries phrased using terminologies selected by the user.
For example, a researcher publishing a research object might assert that the study is reproducible \emph{sensu} Whole Tale.
Another researcher filtering discovered Research Objects by the property NAS::reproducible would find this study
	either if WT::reproducible had been found to imply NAS::reproducible generally, or if other assertions made by the author
	about the Object satisfy the requirements of the latter term in conjunction with the implications of  WT::reproducible.

Going forward, the Whole Tale project aims to explore the various terminologies for reproducibility with the goal of identifying 
	what might be considered to be the principle components of reproducibility in science as a whole.
As a community we could then determine how various terms and definitions can be seen as compositions of these components.
This in turn would reveal how Research Object infrastructure should reason about these terms, and  how claims 
	made in terms of one set of definitions could be converted to claims using another set of definitions.


